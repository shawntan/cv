%!TEX TS-program = xelatex
\documentclass[]{friggeri-cv}
\addbibresource{bibliography.bib}
\usepackage{marvosym} % Allows the use of symbols
\usepackage{amsmath} % Allows the use of symbols
\usepackage{enumitem}
\usepackage{etoolbox}
\providetoggle{short}
\settoggle{short}{true}
\begin{document}
\header{shawn}{tan}
{aspiring computer scientist

}
% In the aside, each new line forces a line break

\begin{aside}
	\section{online}
	\Email~ \href{mailto:shawn@wtf.sg}{shawn@wtf.sg}
	\href{https://blog.wtf.sg}{http://blog.wtf.sg}
	\href{https://scholar.google.ca/citations?user=57Nf7EYAAAAJ&hl=en}{google.scholar}
	\href{http://github.com/shawntan}{shawntan@github}
	\href{http://sg.linkedin.com/in/tanshawn}{tanshawn@linkedin}
	\section{languages}
	english (proficient)
	mandarin
	\section{programming}
	{\color{red} $\varheartsuit$} Python, JavaScript, Java, Ruby, C, Perl
	\section{technical skills}
	HTML \& CSS, \LaTeX,
	Linux System Administration
\end{aside}
\begin{tabular}{ p{0.45cm} p{8cm} p{0.45cm} p{6.5cm} }
	\Male 	& Tan Jing Shan, Shawn	 & \Sagittarius &  DOB: 1987/12/17\\
	\Letter & 3601 rue Sainte Famille, \# 1610, QC H2X 2L6 & \Mobilefone & (+1) 
	514 562-7375\\
\end{tabular}


\iftoggle{short}{}%
{
\section{about}
I'm currently a PhD student at the University of Montreal, where I work at the 
Montreal Institute of Learning Algorithms (MILA). \iftoggle{short}{}{I enjoy 
working on machine learning projects, and keeping up with the latest in neural 
networks research. I maintain a \href{http://blog.wtf.sg}{blog} where I write 
about my machine learning projects.}

}
\section{research interests}
natural language processing, compositional generalisation, graphical models, 
amortised variational inference, structured prediction
\section{experience}

\begin{entrylist}
	\entry
	{2016-present}
	{\href{https://mila.quebec}{Montreal Institute for Learning Algorithms}}
	{Student}
	{Exploring better deep learning architectures and structured prediction 
	methods for modelling natural language.}

	\entry
	{2014-2016}
	{Speech Recognition Group, National University of Singapore}
	{Research Assistant}
	{Working on the machine learning aspects of speech recognition using neural 
	networks, with applications to speaker adaptation and noise robustness.}
	\entry
	{2013-2014}
	{\href{http://semantics3.com}{Semantics3}}
	{Software/Data Engineer}
	{
	\iftoggle{short}{Data scientist / Machine learning engineer related tasks 
	for processing data.}%
	{
		\begin{itemize}[itemsep=0pt,topsep=0pt]
			\item Classification of products into a hierarchical taxonomy using 
				a cascading naive bayes model built using the Lucene index.
			\item Name equivalency model used as part of the disambiguation 
				process to determine if two product names are referring to the 
				same thing.
			\item Experimented with using conditional random fields for 
				attribute extraction from product names.
		\end{itemize}
	}
	}
	\entry
	{2010-2012}
	{Web Information Retrieval and NLP Group (WING), NUS}
	{Undergraduate researcher}
	{Did two undergraduate research projects}
	\entry
	{2008-2009}
	{Singapore Armed Forces}
	{Manpower Officer}
	{Administrative position managing National Servicemen}
\iftoggle{short}{}%
{
	\entry
	{01-06 2007}
	{\href{http://www.np.edu.sg/ict/facilities/rhymes/Pages/loc\_rhymes.aspx}{RHyMeS 
	Center, Ngee Ann Polytechnic}}
	{Research/Teaching Assistant}
	{
		\begin{itemize}[itemsep=0pt,topsep=0pt]
			\item Provided technical support for students working on the RHyMeS 
				project.
			\item Facilitated workshops on using the RHyMeS SDK and API.
			\item  Taught students working on their projects how to use the Java 
				Swing UI API
		\end{itemize}
	}
}
\end{entrylist}
\section{education}
\begin{entrylist}
	\entry
	{2018-present}
	{PhD in Computer Science}
	{University of Montreal}
	{}


	\entry
	{2016-2018}
	{Masters in Computer Science}
	{University of Montreal}
	{}

	\entry
	{2009-2012}
	{Bachelors of Computing (BComp)}
	{National University of Singapore}
	{\iftoggle{short}{}{%
		Major in Computer Science - Upper 2$^{\text{nd}}$ Class Honours \\
		Special Programme in Computing (Turing Programme) \\
		Focus Area: Artificial Intelligence\\
		CAP: 4.39 / 5
		}
	}
\entry
	{2004-2007}
	{Diploma in Information Technology}{Ngee Ann Polytechnic, Singapore}
	{\iftoggle{short}{}{%
		Specialisation in Software Engineering - Diploma with Merit\\
		GPA: 3.8 / 4
	}}
	\iftoggle{short}{}%
	{
	\entry
	{2003}
	{GCE 'O' Levels}{Mayflower Secondary School, Singapore}
	{}}
\end{entrylist}

\section{publications}
\printbibsection{inproceedings}{conference papers}
\printbibsection{report}{reports}
\printbibsection{article}{others}

%\section{projects}
%\begin{entrylist}
%	\entry{10 2014}
%	{
%		\href
%		{https://github.com/shawntan/neural-turing-machines}
%		{Neural Turing Machines implementation in Theano}
%	}
%	{}
%	{
%		Implementation of system described in 
%		\href{http://arxiv.org/abs/1410.5401}{"Neural Turing Machines." by Alex 
%		Graves, Greg Wayne, and Ivo Danihelka.}
%	}
%\iftoggle{short}{}%
%{
%	\entry{2012}
%	{
%		\href
%		{http://wing.comp.nus.edu.sg/portal/publications.html?view=publication&task=show&id=178}
%		{Predicting Web 2.0 Thread Updates}
%	}
%	{}
%	{
%		A method to estimate arrival times of new posts to forum discussion 
%		threads and an evaluation metric for measuring the effectiveness of an 
%		incremental crawler of a site with time-sensitive data.
%	}
%	\entry{04 2012}
%	{
%		\href
%		{https://github.com/shawntan/mm-crawl}
%		{Focused Web Crawling using Markov Decision Processes}
%	}
%	{}
%	{
%		Using the least-squares policy iteration algorithm and some heuristics	
%		to perform focused crawling. For CS4246R AI for Planning and Decision 
%		Making (Research Project)
%	}
%
%
%	\entry{2010 - 2011}
%	{\href{http://wing.comp.nus.edu.sg/portal/publications.html?view=publication&task=show&id=151}{grab\textit{smart}: 
%	A User-centric Web Information Extraction System}}
%	{}
%	{
%		An integrated system that allows users to easily select portions of a 
%		page to extract, and subsequently extracts the data in a manner that is 
%		robust to site layout changes.
%	}
%	\entry{04 2011}
%	{
%		\href
%		{https://github.com/shawntan/lspi-tetris-agent}
%		{LSPI Tetris Agent}
%	}
%	{}
%	{
%		Application of the least-squares policy iteration algorithm to play 
%		Tetris as part of the CS3243 Introduction to Artificial Intelligence 
%		course. The agent can complete 100,000 lines on average, and up to a 
%		maximum of about a million lines.
%	}}
%\end{entrylist}
%
%
%\iftoggle{short}{}%
%{
%\section{organisations}
%\begin{entrylist}
%	\entry
%	{2010-2012}
%	{NUS Hackers}
%	{Coreteam Member (Member of Executive Committee)}
%	{
%		Duties include facilitating NUS Hacker activities: Hack \& Roll 2012, 
%		Friday Hacks (a weekly tech talk), and maintaining the 
%		download.nus.edu.sg mirror service.
%	}
%	\entry
%	{2003-2012}
%	{Singapore Scouts Association}{Assistant Venture Scout Leader}
%	{
%		Attached to the Mayflower Secondary School Boy Scouts Group. Duties a	
%		include facilitating camps, weekly meetings, and mentoring both scouts 
%		and ventures.
%	}
%\end{entrylist}}


\end{document}
